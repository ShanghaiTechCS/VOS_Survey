\section{Discussion}

\subsection{Evaluation Metrics}
In a supervised evaluation framework, given a groundtruth mask $G$ on a particular frame and an output segmentation $M$,
any evaluation measure ultimately has to answerthe question how well $M$ fits $G$. As justified in \cite{pont2016supervised}, 
for images one can use two complementary points of view, regionbased and contour-based measures. As videos extends the
dimensionality of still images to time, the temporal stability of the results must also be considered.

\subsubsection{Accuracy}
\paragraph{Region Similarity $\J$}
To measure the region-based segmentation similarity, i.e. the number of mislabeled pixels,
we employ the Jaccard index $\J$ defined as the intersectionover-union of the estimated segmentation and the groundtruth mask.
The Jaccard index has been widely adopted since its first appearance in PASCAL VOC2008\cite{martin2004learning}, 
as it provides intuitive, scale-invariant information on the number of mislabeled pixels. Given an output segmentation $M$ and 
the corresponding ground-truth mask $G$ it is defined as $\J = \frac{|M∩G|}{|M∪G|}$

\paragraph{Contour Accuracy $\F$}
From a contour-based perspective, one can interpret $M$ as a set of closed contours $c(M)$
delimiting the spatial extent of the mask. Therefore, one
can compute the contour-based precision and recall $Pc$ and
$Rc$ between the contour points of $c(M)$ and $c(G)$, via a
bipartite graph matching in order to be robust to small inaccuracies, as proposed in \cite{martin2004learning}.

We consider the so called F-measure $F$ as a good trade-off between the two, defined
as $F = \frac{2PcRc}{Pc+Rc}$.

\paragraph{Temporal Stability $\T$}
Temporal stability $\T$ . Intuitively, $\J$ measures how well the pixels of the two masks match, while $F$ measures the
accuracy of the contours. However, temporal stability of the results is a relevant aspect in video object segmentationsince the evolution of object shapes is an important cue for
recognition and jittery, unstable boundaries are unacceptable in video editing applications. 

\subsubsection{Execution Time}

\subsection{Results}

% run the experiment on all datasets

\subsubsection{Single Object Semi-supervised VOS}

\subsubsection{Multiple Objects Semi-supervised VOS}

\subsubsection{Unsupervised VOS}

\begin{table*}[t!h]
	\begin{center}
%		\setlength\tabcolsep{5pt}
		\begin{tabular}{|c|c|c|c|c|c|c|c|c|c|}
			\hline
			&NLC~\cite{Faktor2014Video} &CUT~\cite{Keuper2015Motion} &FST~\cite{Papazoglou2013Fast} &SFL~\cite{Cheng2017SegFlow:} &LMP~\cite{Tokmakov2017Learninga} &FSEG~\cite{Jain2017FusionSeg:} &LVO~\cite{Tokmakov2017Learning} & ARP~\cite{Koh2017Primary} 
			&IET~\cite{Li2018Instance} \\
			\hline
			$\J$ Mean &55.1 &55.2	&55.8 &67.4 & 70.0 & 70.7 &75.9 &76.2 &78.5 \\
			\hline
			$\F$ Mean &52.3 &55.2 &51.1 &66.7 &65.9 &65.3 &72.1 &70.6 &75.5 \\
			\hline
		\end{tabular}
	\end{center}
	\caption{The result of unsupervised methods on the DAVIS 2016 validation set.}
	\label{table:davis_unsuperivsed}
\end{table*}

\subsection{Summary}

\subsection{Future Research Directions}