\documentclass[10pt,twocolumn,letterpaper]{article}

\usepackage{cvpr}
\usepackage{times}
\usepackage{epsfig}
\usepackage{graphicx}
\usepackage{amsmath}
\usepackage{amssymb}
\usepackage{epstopdf}
\usepackage{multirow}
\usepackage{bbding}
\usepackage{colortbl}
\usepackage{subfigure}
\usepackage{float}


% Include other packages here, before hyperref.
\graphicspath{{figures}}
\newcommand{\argmax}{\operatornamewithlimits{argmax}}
\newcommand{\mtodo}[1]{{\bf \textcolor{blue}{[TODO: #1]}}}
\newcommand{\rev}[1]{{\textcolor{blue}{#1}}}
\newcommand{\dfun}[1]{{\llbracket #1 \rrbracket}}

% -------- Measure names --------
\newcommand{\J}{\mathcal{J}}
\newcommand{\F}{\mathcal{F}}
\newcommand{\T}{\mathcal{T}}

% If you comment hyperref and then uncomment it, you should delete
% egpaper.aux before re-running latex.  (Or just hit 'q' on the first latex
% run, let it finish, and you should be clear).
\usepackage[breaklinks=true,bookmarks=false]{hyperref}

\cvprfinalcopy % *** Uncomment this line for the final submission


%\def\cvprPaperID{****} % *** Enter the CVPR Paper ID here
\def\httilde{\mbox{\tt\raisebox{-.5ex}{\symbol{126}}}}

% Pages are numbered in submission mode, and unnumbered in camera-ready
%\ifcvprfinal\pagestyle{empty}\fi
\setcounter{page}{1}
\begin{document}

\title{Video Object Segmentation: A Survey}

\author{Zhixin Piao \quad Yongfei Liu \quad Qiong Huang \\
\small{School of Information Science and Technology, ShanghaiTech University}\\
{\tt\small \{piaozhx,liuyf,huangqiong\}@shanghaitech.edu.cn}
}

\maketitle

\begin{abstract}
Video object segmentation is more and more being of interest for computer vision and machine learning researchers.
Many applications on the rise need accurate and efficient segmentation mechanisms: autonomous driving, indoor navigation, 
and even virtual or augmented reality systems to name a few.
This demand coincides with the rise of deep learning approaches in almost every field or application target related to computer vision,
including video objects segmentation or scene understanding.

This paper provides a review on deep learning method for video objects segmentation in various setting, which can mainly be divided in semi-supervised and unsepervised manner. The huge difference between 
two setting is whether the first frame groundtruth is given in test phase.
Firstly, we address the important of video objects for many practical applications. Then, the popular deep learning archtectures and datasets are exposed  to help researchers have a big picture of 
video objects segmentation and an understanding of the efforts worked by former researchers.
Then we revise some existing methods which can tackle video objects segmenation successfully, and highlight their contributions and significance.
Finally, quantitative results are given for the described methods and the datasets in which they were evaluate, 
following up with a discussion of the results. 
At last, we point out a set of promising future works and draw our own conclusions about state-of-art video objects segmenation algorithms.



\end{abstract}

\section{Introduction}
Nowadays, semantic segmentation applied to still $2D$ images, video, and even $3D$ or volumetirc data is one of the key problems in the field of computer vision.
Especially, video object segmentation is one of the high-level task, which can greatly pave the ways toward complete video scene understanding.
Recently, increasing attention are paid in autonomous driving\cite{geiger2012we, cordts2016cityscapes, ess2009segmentation}, which is regarded as the most challenging part in visual understanding. 
Such problem has been addressed in the past using various traditional computer vision techniques, which cannot slove the problem clearly because of large amounts of video sequence data.
The deep learning revolution has turned the tables so that many computer vision problems
video object segmentation among them are being tackled using deep architectures, usually Convolutional Neural Networks (CNNs)\cite{farabet2013learning, ning2005toward}, whicch have excellcent 
performance in terms of accuracy and even efficiency.

In last few years, video object segmentation make an impressive progress due to the existence of excellent large scale video datesets \cite{DAVIS2016, SegTrack, Youtube}, which provide a densely
accurate labeling groundtruth. A lots of work mainly address problems on video instance segmentation and primary object segmentation in DAVIS challenges setting. So this survey mainly revise some 
excellent work on recent popular video objects  segmentation topic.

To the best of our knowledge, this is the first review to focus on deep learning for video objects segmentation mainly based on DAVIS challenges setting. In contrast to this survey, there lots of 
semantic segmentation survey, which mainly apply the object segmentation on still $2D$ images. Howerver, video sequences have abundunt information in temporal diemension. Many techniques can utilize
the more motion cues besides apperances cues. So the most methods are introduced here that combine spatial-temporal information together, which will give greatly benefits to tackle video objects 
segmentation. 

The key contributions of our work are as follows:
\begin{enumerate}
    \item  We provide a broad survey of existing network backbones and datasets that might be useful for segmentation projects with deep learning techniques.
    \item  An in-depth and organized review of the most significant methods that use deep learning and combine spatial-temporal information for video objects segmentation, 
           their origins, and their contributions.
    \item  A thorough performance evaluation, which include the accuracy and bounary accuracy.
    \item  A discussion about the aforementioned results, as well as a list of possible future works that might set the course of upcoming advances,
           and a conclusion summarizing the state of the art of the field.
    
\end{enumerate}

The remainder of this paper is organized as follows. Firstly, Section $2$ introduces the video objects segmentation problem as well as notation and conventions commonly used in the literature.
Other background concepts such as common deep neural networks and several specially techniques for videos are also reviewed. Next, Section 3 describes existing datasets and corresponding attributes.
section $4$ review some methods tackling video sequence based on different setting, semi-supervised and unsupervised ways. This section focus on descirbing the the contributions in papers and highlights some
specific techniques when dealing with videos. Finally, Section 5 presents a brief discussion on the presented methods based on their quantitative results on the aforementioned datasets. 
In addition, future research directions are also laid out. At last, Section 6 summarizes the paper and draws conclusions about this work and the state of the art of the field.
\section{Backgrounding}


\subsection{Common Deep Network BackBone}
Deep convolutional neural networks (CNNs) \cite{Lecun2014Backpropagation} have been widely used in light of their successes in many computer vision tasks, such as, face recognition\cite{Schroff2015FaceNet},  object detection\cite{Ren2015Faster}, and image classification\cite{Krizhevsky2012ImageNet}. CNNs use the multi-layer nonlinear function to approximate the mapping function from input to output.

The famous LeNet contains 5 layers. Later as the enhancement in the computing capability of GPUs, deeper CNNs with more convolutional layers, including AlexNet (8 layers) \cite{Krizhevsky2012ImageNet} and VGGNet (16 layers) (16 layers)\cite{Simonyan-VGG} are proposed. Experimental results show that the increase in depth improves the capability of CNNs for image classification. Fig.\ref{VGG} shows the architecture of VGG.

\begin{figure}[ht]
    \centering
    \includegraphics[width=0.5\textwidth]{./figure/VGG.png}
    \caption{VGG Network}
    \label{VGG}
\end{figure}

In GoogLeNet\cite{Szegedy-GoogLeNet}, Szegedy \etal propose the `\emph{Inception module}', which is aimed to find out an optimal local sparse structure in a convolutional network. The Inception architecture vision contains three paths with filters sizes $1 \times 1$, $3 \times 3$ and $5\times 5$, an alternative parallel pooling path, and a combination of all those layers with their output filter banks concatenated into a single output vector forming the input of the next stage.
Such an Inception network is a network consisting of modules of such type stacked upon each other, with occasional max-pooling layers with stride 2 to decrease the resolution of the grid.
Compared with AlexNet \cite{Krizhevsky2012ImageNet}, VGGNet\cite{Simonyan-VGG}, GoogLeNet\cite{Szegedy-GoogLeNet}, ResNet is a more deeper framework with residual learning strategy designed by He \etal in \cite{He-Resnet}. ResNet is neural networks in which each layer consists of a residual module $f_i$ and a skip connection bypassing $f_i$.
In \cite{veit2016residual}, Veit \etal propose a new interpretation of residual networks, which shows residual networks can be seen as a collection of many paths of differing length.
In \cite{xie2017aggregated}, Xie \etal propose a highly modularized network architecture named `\emph{ResNeXt}' and introduces a dimension `\emph{cardinality}' as a factor to the dimensions of depth and width. A module in ResNeXt performs a set of transformations each on a low-dimensional embedding, whose outputs are aggregated by summation and the transformations to be aggregated are all of the same topology.
In \cite{huang2016densely}, Huang \etal design the Dense Convolutional Network (DenseNet), which connects each layer
to every other layer in a feed-forward way. For each layer, the feature-maps of all preceding layers are used as inputs, and its own feature-maps are used as inputs into all subsequent layers.

\begin{figure}[ht]
    \centering
    \includegraphics[width=0.5\textwidth]{./figure/FCN.png}
    \caption{FCN framework}
    \label{FCN}
\end{figure}

\subsection{Common Deep Network Architecture}
\subsubsection{FCN}
Semantic pixel-wise segmentation is an active topic of research. Recently, CNNs also have been introduced to solve segmentation problem. 

In \cite{Long2015Fully}, Long \etal define and detail the space of fully convolutional networks, which take input of arbitrary size and produce correspondingly-sized output with efficient inference and learning. A skip architecture is also designed that combines semantic information from a deep, coarse layer with appearance information from a shallow, fine layer to produce accurate and detailed segmentations. 


\subsubsection{SegNet}
In \cite{SegNet}, Badrinarayanan \etal present a deep fully convolutional neural network architecture for semantic pixel-wise segmentation termed SegNet, which consists of an encoder network, a corresponding decoder network followed by a pixel-wise classification layer. The role of the decoder network is to map the low resolution encoder feature maps to full input resolution feature maps for pixel-wise classification. Specifically, the decoder uses pooling indices computed in the max-pooling step of the corresponding encoder to perform non-linear upsampling.

\begin{figure}[ht]
    \centering
    \includegraphics[width=0.5\textwidth]{./figure/SegNet.png}
    \caption{SegNet framework}
    \label{SegNet}
\end{figure}

\subsubsection{U-Net}
Ronneberger \etal propose U-Net for biomedical image segmentation in \cite{Ronneberger2015U}. The architecture of U-Net consists of a contracting path to capture context and a symmetric expanding path that enables precise localization. 
In the upsampling part, U-Net has also a large number of feature channels, which allow the network to propagate context information to higher resolution layers. As a consequence, the expansive path is more or less symmetric to the contracting path, and yields a u-shaped architecture.

\begin{figure*}
    \begin{center}
        \includegraphics[width=\textwidth]{figure/unet.png}
    \end{center}
    \caption{U-Net framework}
    \label{unet}
\end{figure*}
\subsection{Common Method for Video}

\subsubsection{Optical Flow}
Video object segmentation is challenging due to fast moving objects, deforming shapes, and cluttered backgrounds. optical flow can be used to propagate an object segmentation over time but, unfortunately, flow is often inaccurate, particularly around object boundaries. Such boundaries are precisely where we want our segmentation to be accurate. To obtain accurate segmentation across time.  Tsai \etal propose an efficient algorithm\cite{OFL}  that considers video segmentation and optical flow estimation simultaneously, formulate a principled, multiscale, spatio-temporal objective function that uses optical flow to propagate information between frames. For optical flow estimation, particularly at object boundaries.they show that segmentation results can be used to refine the optical flow, and vice versa, in the proposed object flow algorithm, and can be efficiently computed by iterative optimization.


\subsection{Common Method for One-Shot}

\subsubsection{Fine-Tune}
Let us assume that one would like to segment an object in a video, for which the only available piece of information is its foreground/background segmentation in one frame. Intuitively, one could analyze the entity, create a model, and search for it in the rest of the frames. For humans, this very limited amount of information is more than enough, and changes in appearance, shape, occlusions, etc. do not pose a significant challenge, because we leverage strong priors: first “It is an object,” and then “It is this particular object.” Our method is inspired by this gradual refinement.

We train a Fully Convolutional Neural Network (FCN) for the binary classification task of separating the foreground object from the background. We use two successive training steps: First we train on a large variety of objects, offline, to construct a model that is able to discriminate the general notion of a foreground object, i.e., “It is an object.” Then, at test time, we fine-tune the network for a small number of iterations on the particular instance that we aim to segment, i.e., “It is this particular object.” The overview of our method is illustrated in Figure 2.

\subsubsection{Data Dreaming}
To train the function f one would think of using ground truth data for M t−1 and M t (like [9], [11], [12]), however such data is expensive to annotate and rare. [9] thus trains on a set of 30 videos (∼ 2k frames) and requires the model to transfer across multiple tests sets. [10] side-steps the need for consecutive frames by generating synthetic masks M t−1 from a saliency dataset of ∼ 10k images with their corresponding mask M t . We propose a new data generation strategy to reach better results using only ∼100 individual training frames.

Ideally training data should be as similar as possible to the test data, even subtle differences may affect quality (e.g. training on static images for testing on videos under-performs [62]). To ensure our training data is in-domain, we propose to generate it by synthesizing samples from the provided annotated frame (first frame) in each target video. This is akin to “lucid dreaming” as we intentionally “dream” the desired data by creating sample images that are plausible hypothetical future frames of the video. The outcome of this process is a large set of frame pairs in the target domain (2.5k pairs per annotation) with known optical flow and mask annotations, see Figure 5.

\subsubsection{Mask Warp}
We use flow in two complementary ways. First, to obtain a better initial estimate of M t we warp M t−1 using the flow F t : M t = f I (I t , w(M t−1 , F t )); we call this "mask warping". Second, we use flow as a direct source of information about the mask M t . As can be seen in Figure 2, when the object is moving relative to background, the flow magnitude kF t k provides a very reasonable estimate of the mask M t . We thus consider using a convnet specifically for mask estimation from flow: M t = f F (F t , w(M t−1 , F t )), and merge it with the image-only version by naive averaging
\section{Datasets}

\subsection{Datasets}
\paragraph{Youtube-Objects~\cite{prest2012learning}} 

\paragraph{DAVIS~\cite{Perazzi2016,PontTuset2017,ponttuset2018}}

\paragraph{SegTrack~\cite{li2013video}}

\section{Methods}

\subsection{Semi-supervised VOS}

\subsubsection{OSVOS}

\subsubsection{LucidTracker}


\subsubsection{PML}


\subsubsection{DyeNet}


\subsubsection{CTN}


\subsection{Unsupervised VOS}
Video object segmentation is the task of extracting spatio-temporal regions that correspond to object moving in at
least one frame in the video sequence. In contrast to semi-supervised VOS, unsupervised video object segmentation have 
more challages and is more practical in real world. In real world case, with the limitation of resources and the diversity of scenario,
it is difficult to simulate various outdoor scenario and collect dataset from it in laboratory. So unsupervised video object 
segmentation have more important impact on our real daily life. To better understand to unsupervised setting, here we list some
obvious difference between semi-supervised video segmentation.
\begin{enumerate}
    \item no first frame groundtruth is provided in test phase.
    \item no finetune process in test phase is needed.
\end{enumerate}

In unsupervised VOS tasks, we can regard them as zero-shot video objects segmentation. Because we cannot have any objects priors
in test phase, namely that the objects in testset does not exit in training phase. The task's main challages is that we need to infer
the primary objects which are moving in video frames automatically. The algorithm can discovers the most salient, or primary, objects,
that move against a video's background or display different color statistics. To better capture the moving objects against the background,
the object motion is the critical cue for identify salients objects throught entire video sequences. Next, we will introduce some important
methods which is commonly used in unsupervised VOS.

\subsubsection{Motion in Video Sequences}
In traditional static image semantic segmentation, the apperance information play an import role, which means that 
the performance is enough good if we can extract more reliable apperance features. But in video setting, thera are various
difficult challlenges caused by object moving, which are motion blur and ambigious and occluated. We just rely on apperance information
, which can  fail in this specifical scenario. Motion information can greatly help reduce this ambigious. We can capture more temporal information
to help locate objects in videos frames.

Jain $et.al$ \cite{Jain2017FusionSeg} propose an end-to-end learning framework for segmenting generic objects in video,
which learns to combine appearance and motion information to produce pixel level segmentation masks for all prominent objects.
They design a two-stream fully convolutional neural network which fuses together motion and apperance in a unified framework.

\begin{figure}
    \begin{center}
    \includegraphics[width=0.5\textwidth]{figure/FSEG_NET.png}
    \end{center}
    \caption{FSEG Network}
    \label{FSEG}
\end{figure}

As shown in Fig \ref{FSEG}, the network can take two different inputs, which are raw images and optical flow images respectively.
The motion branch can map the motion into foreground objects, which can greatly capture temporal infomation.
In the last fusion stage, this method does not simply concat two stream feature to get the final prediction.
They design a new fusion strategy to create three independent parallel branches. They apply $1\times1$ convolution to apperance and
motion branch. Finally they apply a layer that thkes the elements-wise maximum to obtain the final prediction. The movivation is that 
an object segmentation prediction is reliable if 1) either apperance or motion model along predicts the object 
segmentation with very strong confidence or 2) their combination together predicts the segmentation with high confidence. 

Besides the two stream strategy, there are still the other fusion strategy to fuse the motion information into apperance 
features to provide additional infomation for building a strong representation of objects that evolves over time. 

\begin{figure}[ht]
    \centering
    \includegraphics[width=0.5\textwidth]{./figure/LVO_NET.png}
    \caption{LVO Network}
    \label{LVO}
\end{figure}

Pavel $et.al$\cite{Tokmakov2017Learning} proposed a new network structures. In Fig\ref{LVO}, they use a motion network to extract motion features, which 
take as side information to get the final prediction. The motion network is a pretrained network.

We can improve the performance by combining apperance feature and motion features. However, how can we get good motion
prediction by inputing optical flow images. Pavel $et.al$\cite{LMPV} proposed a encoder-decoder style to estimate the motion 
of video directly.

\begin{figure}[ht]
    \centering
    \includegraphics[width=0.5\textwidth]{./figure/LMP_NET.png}
    \caption{LMP Network}
    \label{LMP}
\end{figure}

In Fig.\ref{LMP}, the input is the optical flow and the output is the moving probability of every single pixel. They use the U-net structure, which
add some shortcut connection to add the coarse information to high level feature, which will greatly help enhance the feature representation power.
After getting the motion map, we can directly use this to predict foreground objects in frames. 

\begin{figure}[ht]
    \centering
    \includegraphics[width=0.5\textwidth]{./figure/LMP_results.png}
    \caption{LMP Segmentation Pipeline}
    \label{LMP_results}
\end{figure}

Fig.\ref{LMP_results} is the whole pipeline using motion map to predict the final foreground objects. Each row shows: (a) raw images,
(b) optical flow estimated with LDOF \cite{brox2009large}, (c) output of motion network with LDOF flow as input, (d) objectness map computed with
proposals \cite{pinheiro2016learning}, (e) initial moving object segmentation result, (f) refined result with CRF.

We can observe that mapping motion prediction to final foreground segmentation is a realiable methods. The temporal cue can provide more realiable
feature to help do prediciton. So it is critical to advantage the motion cue.


\subsubsection{Visual Memory in Video Sequences}
Motion cue can greatly help capture temporal information. However, motion just can capture the substantial frames and cannot encode more long range
temporal information over several frames. In reality, the RNN have nature attributes that encoder long time dependency. However, the RNN just can 
receive the vector. So representation power is very limited. So Pavel $et.al$\cite{Tokmakov2017Learning} replace the vector in LSTM by the convolution 
operation, which can greatly improve representation power and is more suitable for video segmentation task.
\begin{figure}[ht]
    \centering
    \includegraphics[width=0.35\textwidth]{figure/LVO_CONVRRU.png}
    \caption{convolutional GRU}
    \label{CONVGRU}
\end{figure}

We can see from Fig.\ref{CONVGRU} that they replace the linear combination by convolution operations and a tanh nonlinearity. For frame $t$ in the video
sequence, ConvGRU uses the two stream representation $x_t$ and previous state $h_{t-1}$ to compute the new state $h_t$. The dynamics of this computaiton 
are guided by an update gate $z_t$, a forget gate $r_t$, The states  and the gates are $3$D tensor, and can cahracterizer spatio-temporal pattern in
the video, effectively memorizing which objects move and where they move to. The ConvGRU applies a total of six convolutional opeartions at each time step.
All operation are fully differentiable. So the parameters of the convolution can be trained in an end-to-end fashion wiht back propagation through times.
But there are some drawbacks that the network is memory-comsuming. It's very hard to train. 

In the paper, the author proposed a bidirectional CONVGRU, which can encode motion from the fisrt frame and the last from. This technicial can improve the feature
representation power further. As the expriments shown, the bidirectional CONVGRU bring $5\%$ improvements.

\subsubsection{Hand Craft Methods}

As we all know, deep learning method make huge progress in computer vision. Howerver, there are still a mount of traditional method based on 
hand craft features, which can produce comparable result with deep learning. Sometimes the traditional method is easy to explainable and stable.

Most current methods for unconstrined fg/bg video segmentation are graph-based \cite{Lee2011Key, Papazoglou2013Fast, zhang2013video}. The video
is represented using an Markov Random Field graphical model which consist of a fg/bg data term for each pixel and pairwise
terms between neighboring pixels. The data term is iteratively refined by learning learning color model of the foreground
and background. For computational reasons, the pairwise  terms are typical considered only between adjacent pixels inthe same 
frame and corresponding pixels in adjacent frames(using optical flow).







% \mtodo{Sort by IoU:
% 	% IET~\cite{li2018instance}, 
% 	ARP~\cite{koh2017}, 
% 	LVO~\cite{tokmakov17}, 
% 	FSEG~\cite{jain2017}, 
% 	LMP~\cite{tokmakov2017}, 
% 	SFL~\cite{cheng2017sfl}, 
% 	FST~\cite{papazoglou2013}, 
% 	CUT~\cite{keuper2015}, 
% 	NLC~\cite{faktor2014}, 
% 	MSG~\cite{ochs2011}, 
% 	KEY~\cite{lee2011}, 
% 	CVOS~\cite{taylor2015},
% 	TRC~\cite{fragkiadaki2012}
% }

\section{Discussion}

\subsection{Evaluation Metrics}
In a supervised evaluation framework, given a groundtruth mask $G$ on a particular frame and an output segmentation $M$,
any evaluation measure ultimately has to answerthe question how well $M$ fits $G$. As justified in \cite{pont2016supervised}, 
for images one can use two complementary points of view, regionbased and contour-based measures. As videos extends the
dimensionality of still images to time, the temporal stability of the results must also be considered.

\subsubsection{Accuracy}
\paragraph{Region Similarity $\J$}
To measure the region-based segmentation similarity, i.e. the number of mislabeled pixels,
one employ the Jaccard index $\J$ defined as the intersectionover-union of the estimated segmentation and the groundtruth mask.
The Jaccard index has been widely adopted since its first appearance in PASCAL VOC2008 \cite{martin2004learning}, 
as it provides intuitive, scale-invariant information on the number of mislabeled pixels. Given an output segmentation $M$ and 
the corresponding ground-truth mask $G$ it is defined as $\J = \frac{|M \cap G|}{|M\cup G|}$

\paragraph{Contour Accuracy $\F$}
From a contour-based perspective, one can interpret $M$ as a set of closed contours $c(M)$
delimiting the spatial extent of the mask. Therefore, one
can compute the contour-based precision and recall $Pc$ and
$Rc$ between the contour points of $c(M)$ and $c(G)$, via a bipartite graph matching in order to be robust to small inaccuracies,
as proposed in \cite{martin2004learning}.
So the F-measure $F$ is a good trade-off between two, which is  defined as $F = \frac{2Pc Rc}{Pc+Rc}$.

\paragraph{Temporal Stability $\T$}
Temporal stability $\T$. Intuitively, $\J$ measures how well the pixels of the two masks match, while $F$ measures the
accuracy of the contours. However, temporal stability of the results is a relevant aspect in video object segmentationsince the evolution of object shapes is an important cue for
recognition and jittery, unstable boundaries are unacceptable in video editing applications. 

\subsubsection{Execution Time}

\subsection{Results}

% run the experiment on all datasets

\subsubsection{Single Object Semi-supervised VOS}

\subsubsection{Multiple Objects Semi-supervised VOS}

\subsubsection{Unsupervised VOS}
As shown in Table\ref{table:unsuperivsed_all_dataset}, we can compare the performance of difference algorithm. In generally, the method based on deep learning would excel the the traditional 
methods.The method combined spatial-temporal information can greatly help improve performance in paper\cite{Tokmakov2017Learning}. And in normal case, optical flow method can help capture the 
temporal information, but it is still not powerful enough. The visual memory module applied in temporal can help improve $4.4\%$ in $\J$ evaluation metrics. Let us see some traditional methods,like
\cite{Koh2017Primary,li2018instance}, they can also gain impressive performance in DAVIS dataset because they can design some specifical feature to model sequences data. Li $et.al$ use instance embedding method
to force the algorithm to learn the instance concepts. So they can get excellent performance in instance segmentation setting.

%  \begin{table*}[t!h]
% 	\begin{center}
% 		\setlength\tabcolsep{3pt}
% 		\begin{tabular}{|c|c|c|c|c|c|c|c|c|c|c|}
% 			\hline
% 			Dataset& Metrics&NLC~\cite{Faktor2014Video} &CUT~\cite{Keuper2015Motion} &FST~\cite{Papazoglou2013Fast} &SFL~\cite{Cheng2017SegFlow:} &LMP~\cite{Tokmakov2017Learning} &FSEG~\cite{Jain2017FusionSeg}  &LVO~\cite{Tokmakov2017Learning} & ARP~\cite{Koh2017Primary} &IET~\cite{Li2018Instance}\\
% 			\hline
% 			\multirow{2}{*}{DAVIS} &$\J$ Mean &55.1     &55.2                        &55.8                          &67.4                         &69.7                             &71.5                           &75.9                           &76.2                       &78.5 \\
% 			\cline{2-11}
% 			&$\F$ Mean &52.3 &55.2 &51.1 &66.7 & 66.3 & infea &72.1 &70.6 &75.5 \\
% 			\hline
% 			\multirow{2}{*}{SegTrack} &$\J$ Mean &52.3 &55.2 &51.1 &66.7 &65.9 &61.4 & infea &70.6 &75.5 \\
% 			\cline{2-11}
% 			&$\F$ Mean &52.3 &55.2 &51.1 &66.7 &65.9 &infea &infea &70.6 &75.5 \\
% 			\hline

% 			\multirow{2}{*}{Youtube} &$\J$ Mean &52.3 &55.2 &51.1 &66.7 &65.9 &68.57 &infea &70.6 &75.5 \\
% 			\cline{2-11}
% 			&$\F$ Mean &52.3 &55.2 &51.1 &66.7 &65.9 &infea &infea &70.6 &75.5  \\
% 			\hline
% 		\end{tabular}
% 	\end{center}
% 	\caption{The result of unsupervised methods on the Video Objects Segmentation datasets .}
% 	\label{table:unsuperivsed_all_dataset}
% \end{table*}


% \subsection{Summary}


\begin{table*}[t!h]
	\begin{center}
		\setlength\tabcolsep{3pt}
		\begin{tabular}{|c|c|c|c|c|c|c|}
		\hline
Dataset& Metrics                   &FSEG~\cite{Jain2017FusionSeg}  &LVO~\cite{Tokmakov2017Learning} &LMP~\cite{Tokmakov2017Learning} & POS~\cite{Koh2017Primary} &IET~\cite{li2018instance}\\
\hline
\multirow{2}{*}{DAVIS} &$\J$ mean  &71.51                           &75.9                           &69.7                            &76.3                          &78.5\\
\cline{2-7}
&$\F$ Mean                         &   --                           &72.1                           &66.3                            &71.1                          &75.5\\
\hline
% \multirow{2}{*}{SegTrack} &$\J$Mean&61.4                            &57.3                           &--                              &80                          &--\\
% \cline{2-7}

% &$\F$ Mean                         &  --                            & --                            &--                              &--                          &--  \\
% \hline
% \multirow{2}{*}{Youtube} &$\J$ Mean&68.57                           & --                            &--                              &--                          &-- \\
% \cline{2-7} 
% &$\F$ Mean                         &  --                            & --                            &--                              &--                          &--   \\
% \hline
\end{tabular}
\end{center}

\caption{The result of unsupervised methods on DAVIS datasets.}
\label{table:unsuperivsed_all_dataset}
\end{table*}



\subsection{Summary}
pass

\subsection{Future Research Directions}

\section{Conclusion}
Deep learning has greatly improved the performance of video objects segmentaion tasks. As the previous work shown, firstly we need to design more and more powerful network
to have rich feature representation. In a word , we need to combine the low-level feature and high-level feature. Then temporal information is another important
cue when tackling  video dataset. We will benefit performance improvements by combining temporal information. There are still huge space for us to explore how 
to utilize the temporal information well.

\newpage
{\small
\bibliographystyle{ieee}
\bibliography{reference}
}


\end{document}
